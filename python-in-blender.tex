\documentclass{beamer}
%
% Choose how your presentation looks.
%
% For more themes, color themes and font themes, see:
% http://deic.uab.es/~iblanes/beamer_gallery/index_by_theme.html
%
\mode<presentation>
{
  \usetheme{Warsaw}      % or try Darmstadt, Madrid, Warsaw, ...
  \usecolortheme{default} % or try albatross, beaver, crane, ...
  \usefonttheme{default}  % or try serif, structurebold, ...
  \setbeamertemplate{navigation symbols}{}
  \setbeamertemplate{caption}[numbered]
} 

\usepackage{minted}
\usepackage[english]{babel}
\usepackage[utf8x]{inputenc}
\usepackage{listings}
\usepackage[absolute,overlay]{textpos}

\title[Python in Blender]{Using Python in Blender}
\author{Łukasz Hryniuk}
\date{\today}

\begin{document}

\begin{frame}
  \titlepage
  \centering\href{https://github.com/hryniuk/blender-intro}{github.com/hryniuk/blender-intro}
\end{frame}

\begin{frame}{Outline}
  \tableofcontents
\end{frame}

\section{Introduction}

\begin{frame}{About me and you}

\begin{itemize}
\item About me
\item About you
\item Number of Blender copies?
\end{itemize}

\end{frame}


\begin{frame}{So...}
  \centering \Huge
  \emph{let's switch to Blender}
\end{frame}

\section{Overview of Python in Blender}

\section{Scripting basics}

\section{Animations with Python}

\begin{frame}{Resources}

\begin{itemize}
\item \href{https://docs.blender.org/api/current/}{Official Python API documentation}
\item Blender built-in Templates
\end{itemize}

\end{frame}

\begin{frame}{}
  \centering \Huge
  \emph{Questions, comments and feedback}
\end{frame}

\begin{frame}{}
  \centering \Huge
  \emph{Thank You}
\end{frame}

\end{document}